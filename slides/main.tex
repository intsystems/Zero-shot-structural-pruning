\documentclass[aspectratio=169]{beamer}
\beamertemplatenavigationsymbolsempty
\usecolortheme{beaver}
\setbeamertemplate{blocks}[rounded=true, shadow=true]
\setbeamertemplate{footline}[page number]


\usepackage{graphicx}
\usepackage{amsmath}
\usepackage{amsfonts}
\usepackage{caption}
\usepackage{subcaption}
\usepackage{float}
\usepackage[english,russian]{babel}
\usepackage{xcolor}



\usepackage[utf8]{inputenc}
\usepackage{amssymb,amsfonts,amsmath,mathtext}
%\usepackage{subfig}
\usepackage{tikz}
\usetikzlibrary{arrows.meta}
\usetikzlibrary{quotes}
\usepackage{tikzscale}
\usepackage{scalefnt}
\usepackage{xcolor}
\usepackage[all]{xy} % xy package for diagrams
\usepackage{array}
\usepackage{multicol}% many columns in slide
\usepackage{hyperref}% urls
\usepackage{hhline}%tables
\usepackage{booktabs}
% \usepackage{biblatex}



\def\bw{\mathbf{w}}
\def\balpha{\boldsymbol{\alpha}}
\def\ltrn{\mathcal{L}_{\mathrm{train}}}
\def\lval{\mathcal{L}_{\mathrm{val}}}
\newcommand{\vect}[1]{\boldsymbol{\mathbf{#1}}}


\definecolor{dark_green}{rgb}{0, 0.788, 0}
\definecolor{dark_red}{rgb}{0.9, 0, 0}


% Your figures are here:
\graphicspath{ {fig/} {../fig/} }

\definecolor{ao(english)}{rgb}{0.0, 0.5, 0.0}
\definecolor{bleudefrance}{rgb}{0.19, 0.55, 0.91}

%----------------------------------------------------------------------------------------------------------

\title[\hbox to 56mm{Structural pruning}]{Структурное прореживание моделей глубокого обучения на основе суррогатных моделей}

\author{М.\,О.~Иванов\inst{} \\
\tt{\footnotesize ivanov.mo@phystech.edu }}
\institute{\inst{} Москва, Московский физико-технический институт \\
\textbf{Научный руководитель}: к.ф.-м.н. Бахтеев Олег Юрьевич} \date{2025}
%----------------------------------------------------------------------------------------------------------
\begin{document}
%----------------------------------------------------------------------------------------------------------
\begin{frame}
\thispagestyle{empty}
\maketitle
\end{frame}
%-----------------------------------------------------------------------------------------------------
\begin{frame}{Цель исследования}

\begin{block}{Цель} 
  Предложить метод структурного прореживания нейросетей.
\end{block}

~\\
\begin{block}{Проблема}
  Существующие методы работают послойно. Учёт взаимосвязей между слоями модели является вычислительно сложной задачей.
\end{block}
~\\
\begin{block}{Метод решения}
  Предлагаемый метод основан на построении суррогатной графовой модели, повторяющей структуру исходной нейросети.
\end{block}

\end{frame}

\begin{frame}{Схема предлагаемого решения}
    
    \begin{figure}
        \centering
        \includegraphics[width=0.9\linewidth]{figs/pruning_graph.png}
        Упрощённая схема residual connection. Каждому ребру сопоставляется $\gamma$, по множеству которых граф оптимизируется.
    \end{figure}
\end{frame}

%----------------------------------------------------------------------------------------------------------


\begin{frame}{Постановка задачи}
\begin{itemize}
\item Архитектура модели представляет собой ориентированный ациклический граф $G = (V, E)$. Каждая вершина $v_i \in V$ соответствует слою нейронной сети, и каждое ребро $e_{ij} \in E$ представляет поток данных из выхода слоя $v_i$ до входа $v_j$. 

\item В данной формулировке удаление вершины означает удаление соответствующего слоя из сети, тогда как удаление ребра к устранению зависимости передачи данных между слоями (например, в случае residual connection).

\item Задана выборка $\mathfrak{D} = \mathfrak{D}_\text{train} \cup \mathfrak{D}_\text{val}$. Задана функция потерь $\mathcal{L}$. Пусть $\boldsymbol w$ -- параметры модели до прунинга, а $\boldsymbol w'$ -- параметры модели после. Задача оптимизации
$$\min_{\boldsymbol w' \in \mathbb{R}^k} \; \left| \mathcal{L}(\mathfrak{D} \mid \boldsymbol w') - \mathcal{L}(\mathfrak{D} \mid \boldsymbol w) \right|.$$

\end{itemize}
\end{frame}


\begin{frame}{Предлагаемый метод}
    \begin{itemize}
        \item Пусть задано $\gamma_{ij} \in [0, 1]$ для каждого ребра $e_{ij} \in E$. Также для графа $G$ задана маска рёбер $\boldsymbol M \in \mathbb{Z}_2^{|E|}$:
        $$ (\boldsymbol M)_{ij} = 1 \quad \Leftrightarrow \quad \gamma_{ij} = 0.$$

        \item Тогда оптимизационная задача:
        $$ \gamma = \arg\min\limits_{\gamma \in [0, 1]^{|E|}} \left( \mathsf{E}_{\boldsymbol M} \left\| G(\gamma, \boldsymbol M) - \mathcal{L}(\boldsymbol M)) \right\|^2_2 \right).$$
    \end{itemize}

\end{frame}

\begin{frame}{Постановка эксперимента}
  \begin{itemize}
    \item Цель -- сравнение качества базовых суррогатных моделей для предложенного метода.

    \vspace{10pt}
    
    \item В эксперименте проводится прунинг ResNet-50, фиксируется число удаляемых рёбер -- 15\%.

    \vspace{10pt}
    
    \item В сравнении участвуют следующие базовые методы:
        \begin{small}
        \begin{itemize} 
        \item \textbf{(LinReg)}: Линейная регрессия,
        \item \textbf{(LogReg)}: Логистическая регрессия,
        \item \textbf{(FCN)}: Простая нейросеть.
        \end{itemize}
      \end{small}
  \end{itemize}
  

  
\end{frame}


\begin{frame}{Результаты эксперимента}
  \begin{itemize}
    \item Рассматриваются задачи классификации изображений из датасета CIFAR10 на 2 класса.
    \item  Приводится точность предсказаний метрики AUC на контроле после прунинга.
  \end{itemize}
  \begin{table}
    \centering
    \begin{tabular}{c|c}
    \toprule
    \textbf{Method} & \textbf{MAPE Test} \\ \midrule
    LinReg & 19.6\% \\
    LogReg & 18.7\% \\
    NN & \textbf{16.8\%} \\ \bottomrule
    \end{tabular}
  \end{table}
  Из таблицы видно, что базовые методы решают поставленную задачу, однако точность недостаточна.
  
\end{frame}



\begin{frame}{Заключение}
    \begin{itemize}
      \item Рассмотрена задача структурного прореживания нейросетей.
      \vspace{10pt}
      \item Предложен метод структурного прореживания нейросетей.
      \vspace{10pt}
      \item Проведены базовые эксперименты для сравнения качества суррогатных моделей.
      \vspace{10pt}
      \item Планируется теоретическое обоснование предложенного метода и его имплементация.
    \end{itemize}
\end{frame}

%----------------------------------------------------------------------------------------------------------
\end{document} 